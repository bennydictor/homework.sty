\documentclass[12pt,a4paper,oneside]{article}

\usepackage{amsmath,amssymb,amsthm}
\usepackage[T2A]{fontenc}
\usepackage[utf8]{inputenc}
\usepackage[russian]{babel}
\usepackage[russian]{homework}

\begin{document}
\homework
{Математика. Домашняя работа \No~42}%
{Ивановым Иваном}%
{1 сентября 2016}%

\begin{definitions}
\DefineNaturalNumberSet
\DefineNaturalNumberSetWithIndex
\DefineNChooseK
\end{definitions}

\begin{problem}{\S7 \No250 (2)}

\Statement
Найдите наибольший элемент последовательности:
\begin{equation}
a_n = \frac n {n^2+9}, n \in \nset
\end{equation}

\Solution
Пусть $f(x)$~--- функция действительного числа:
$$
f(x) = \frac x {x^2+9}, x \in \rset
$$
Найдем $x$, при котором $f(x)$ принимает наибольшее значение. Для этого возьмем её производную:
\begin{align*}
f'(x) &= \frac{1\times(x^2+9)-x\times2x}{(x^2+9)^2} \\
f'(x) &= \frac{9-x^2}{(x^2+9)^2}
\end{align*}
И приравняем её к нулю:
\begin{align*}
\frac{9-x^2}{(x^2+9)^2} &= 0 \\
9-x^2 &= 0 \\
x^2 &= 9 \\
x &= \pm 3
\end{align*}
Следовательно, $x = -3$~--- точка минимума $f$, а $x = 3$~--- её точка максимума.
Следовательно, $x = 3$ также является точкой максимума $a_n$.
$$
a_3 = \frac 3 {3^2+9} = \frac 3 {18} = \frac 1 6
$$

\Answer
Наибольший элемент последовательности~--- $a_3 = \frac 1 6$.

\end{problem}
\begin{problem}{Задача про $\comb_n^k$}

\Statement
Докажите, что
$$
{n \choose 0} + {n \choose 1} + \cdots + {n \choose n} = 2^n
$$

\Proof
Бином Ньютона:
$$
(x+1)^n = \sum_{k=1}^n {n \choose k} x^k
$$
Подставим $x = 1$:
\begin{align*}
(1+1)^n = \sum_{k=1}^n {n \choose k} 1^k
2^n = \sum_{k=1}^n {n \choose k}
\end{align*}
\QED

\end{problem}

\end{document}
