\documentclass[12pt,a4paper,oneside]{article}

\usepackage{amsmath,amssymb,amsthm}
\usepackage[english]{babel}
\usepackage{homework}

\begin{document}
\homework
{Mathematics. Homework \No~42}%
{John Smith}%
{September 1\textsuperscript{st} 2016}%

\begin{definitions}
\DefineNaturalNumberSet
\DefineNaturalNumberSetWithIndex
\DefineNChooseK
\end{definitions}

\begin{problem}{\S7 \No250 (2)}

\Statement
Find the largest element of the sequence:
\begin{equation}
a_n = \frac n {n^2+9}, n \in \nset
\end{equation}

\Solution
Let $f(x)$ be function of a real number:
$$
f(x) = \frac x {x^2+9}, x \in \rset
$$
We'll find $x$, for which $f(x)$ is minimal. To do that, take it's derivative:
\begin{align*}
f'(x) &= \frac{1\times(x^2+9)-x\times2x}{(x^2+9)^2} \\
f'(x) &= \frac{9-x^2}{(x^2+9)^2}
\end{align*}
Now set the derivative to zero:
\begin{align*}
\frac{9-x^2}{(x^2+9)^2} &= 0 \\
9-x^2 &= 0 \\
x^2 &= 9 \\
x &= \pm 3
\end{align*}
Hence, $x = -3$ is a local minimum of $f$, and $x = 3$~--- it's local maximum.
$x = 3$ is also a local maximun of $a_n$.
$$
a_3 = \frac 3 {3^2+9} = \frac 3 {18} = \frac 1 6
$$

\Answer
The largest element of the sequence is $a_3 = \frac 1 6$.

\end{problem}
\begin{problem}{A problem about $\comb_n^k$}

\Statement
Prove that
$$
{n \choose 0} + {n \choose 1} + \cdots + {n \choose n} = 2^n
$$

\Proof
Binomial theorem:
$$
(x+1)^n = \sum_{k=1}^n {n \choose k} x^k
$$
Set $x = 1$:
\begin{align*}
(1+1)^n = \sum_{k=1}^n {n \choose k} 1^k
2^n = \sum_{k=1}^n {n \choose k}
\end{align*}
\QED

\end{problem}

\end{document}
